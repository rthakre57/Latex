%Preparing question Paper for examination in latex
\documentclass[a4paper,12pt]{article}   % Document class and font size
\usepackage[utf8]{inputenc}              % Encoding for input
\usepackage{amsmath}                     % Math package for equations
\usepackage{graphicx}                    % For including images
\graphicspath{{C:/Users/rosha/OneDrive/Desktop/}{C:/Users/rosha/OneDrive/Desktop/}}
\usepackage{enumitem}                    % Customizing lists
\usepackage{hyperref}                    % Hyperlinks (optional)
\usepackage{multicol}                    % For multiple columns layout
\usepackage{fancyhdr}                    % For header and footer

% Page layout settings
\usepackage{geometry}
\geometry{a4paper, margin=1in}

% Set the header and footer style
\pagestyle{fancy}
\fancyhf{}
\fancyhead[L]{\textbf{Name of the Institution}}  % Left header
\fancyhead[C]{\textbf{Examination Question Paper}} % Center header
\fancyhead[R]{\textbf{Subject: XYZ}}            % Right header
\fancyfoot[C]{\thepage}                         % Footer with page number

% Title information
\title{Mid-term Examination \\ Subject: XYZ \\ Class: 12 \\ Time: 3 Hours}
\author{Instructor: Your Name}
\date{\today}

\begin{document}
	
	\maketitle                                     % Creates the title
	
	\noindent\textbf{Instructions:}
	\begin{itemize}
		\item The exam consists of 5 sections.
		\item Answer all questions in each section.
		\item Use separate sheets for rough work.
		\item Total marks: 100.
	\end{itemize}
	
	% Section 1: Multiple Choice Questions (MCQs)
	\section*{Section 1: Multiple Choice Questions (20 marks)}
	\noindent\textbf{Answer all questions. Each question carries 1 mark.}
	
	\begin{enumerate}
		\item Which of the following is the correct formula for calculating the area of a circle?
		\begin{enumerate}[label=(\alph*)]
			\item \(A = 2\pi r\)
			\item \(A = \pi r^2\)
			\item \(A = \frac{1}{2}\pi r^2\)
			\item \(A = \pi d^2\)
		\end{enumerate}
		
		\item Which data structure follows the First-In-First-Out (FIFO) principle?
		\begin{enumerate}[label=(\alph*)]
			\item Stack
			\item Queue
			\item Array
			\item Tree
		\end{enumerate}
		
		\item What is the value of \( \int_0^1 x^2 \, dx \)?
		\begin{enumerate}[label=(\alph*)]
			\item \( \frac{1}{3} \)
			\item \( \frac{1}{2} \)
			\item \( 1 \)
			\item \( 0 \)
		\end{enumerate}
	\end{enumerate}
	
	\vspace{0.5cm}
	
	% Section 2: Short Answer Questions
	\section*{Section 2: Short Answer Questions (30 marks)}
	\noindent\textbf{Answer all questions. Each question carries 5 marks.}
	
	\begin{enumerate}
		\item Describe the difference between RAM and ROM.
		\item Explain the working of a binary search algorithm.
		\item Define and give an example of a recursive function.
		\item Differentiate between TCP and UDP.
		\item Explain the laws of thermodynamics briefly.
	\end{enumerate}
	
	\vspace{0.5cm}
	
	% Section 3: Long Answer Questions
	\section*{Section 3: Long Answer Questions (30 marks)}
	\noindent\textbf{Answer any 3 questions. Each question carries 10 marks.}
	
	\begin{enumerate}
		\item Describe the architecture of the OSI model in detail.
		\item Solve the differential equation: \( \frac{dy}{dx} + 2xy = 0 \).
		\item Discuss the impact of Artificial Intelligence on modern industries.
		\item Write an essay on the ethical considerations of data privacy.
	\end{enumerate}
	
	\vspace{0.5cm}
	
	% Section 4: Numerical Problems
	\section*{Section 4: Numerical Problems (20 marks)}
	\noindent\textbf{Solve any 2 problems. Each problem carries 10 marks.}
	
	\begin{enumerate}
		\item Find the area under the curve for \(f(x) = x^3\) between \(x = 0\) and \(x = 2\).
		\item A car accelerates from rest with an acceleration of 3 m/s\(^2\). How far does it travel in 10 seconds?
		\item Solve for \(x\) in the quadratic equation \(x^2 - 5x + 6 = 0\).
	\end{enumerate}
	
	\vspace{0.5cm}
	
	% Section 5: Diagram-based Questions
	\section*{Section 5: Diagram-based Questions (Optional)}
	\noindent\textbf{Answer the following questions based on the diagram provided.}
	
	\begin{enumerate}
		\item Label the parts of the neuron structure in the diagram below and explain their functions.
		\begin{center}
			\includegraphics[width=0.5\textwidth]{photo1.png}  % Add image
		\end{center}
		
		\item Identify the components of the circuit diagram provided below.
		\begin{center}
			\includegraphics[width=0.5\textwidth]{photo2.png}  % Add image
		\end{center}
	\end{enumerate}
	
\end{document}
