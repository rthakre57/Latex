%Beamer Presentation on Intellectual Property Rights
\documentclass{beamer}
\usetheme{Madrid}  % Beamer theme for the presentation
\usepackage{graphicx} % For including images
\usepackage{hyperref} % For hyperlinks

% Title Information
\title{Intellectual Property Rights (IPR)}
\author{Your Name}
\institute{Your Institution}
\date{\today}

\begin{document}
	
	% Title Slide
	\begin{frame}
		\titlepage
	\end{frame}
	
	% Table of Contents
	\begin{frame}{Outline}
		\tableofcontents
	\end{frame}
	
	% Slide 1: Introduction to Intellectual Property Rights
	\section{Introduction}
	\begin{frame}{What are Intellectual Property Rights?}
		\begin{itemize}
			\item Intellectual Property Rights (IPR) are the rights given to persons over the creations of their minds.
			\item These rights usually give the creator an exclusive right over the use of their creation for a certain period.
			\item Types of IPR include copyrights, patents, trademarks, and trade secrets.
		\end{itemize}
		\vspace{0.5cm}
		\textbf{Why is IPR important?}
		\begin{itemize}
			\item Encourages innovation and creativity.
			\item Provides economic incentives for creators.
			\item Ensures legal protection against infringement.
		\end{itemize}
	\end{frame}
	
	% Slide 2: Types of Intellectual Property Rights
	\section{Types of Intellectual Property Rights}
	\begin{frame}{Types of IPR}
		\begin{enumerate}
			\item \textbf{Copyrights:}
			\begin{itemize}
				\item Protects original literary, artistic, and musical works.
				\item Duration: Life of the author plus 70 years.
			\end{itemize}
			\item \textbf{Patents:}
			\begin{itemize}
				\item Protects inventions and innovations.
				\item Grants exclusive rights to the patent holder for 20 years.
			\end{itemize}
			\item \textbf{Trademarks:}
			\begin{itemize}
				\item Protects symbols, logos, and brand names used to distinguish goods and services.
				\item Renewable every 10 years.
			\end{itemize}
			\item \textbf{Trade Secrets:}
			\begin{itemize}
				\item Protects confidential business information.
				\item No formal registration required, but legal protection is available.
			\end{itemize}
		\end{enumerate}
	\end{frame}
	
	% Slide 3: Copyright
	\section{Copyright}
	\begin{frame}{What is Copyright?}
		\begin{itemize}
			\item Copyright protects original works of authorship such as literary, dramatic, musical, and artistic works.
			\item The rights granted include the right to reproduce, distribute, perform, and display the work publicly.
			\item Copyright automatically applies once the work is created and fixed in a tangible form.
		\end{itemize}
		\vspace{0.5cm}
		\textbf{Examples:}
		\begin{itemize}
			\item Books, music, paintings, films, software.
		\end{itemize}
	\end{frame}
	
	% Slide 4: Patents
	\section{Patents}
	\begin{frame}{What is a Patent?}
		\begin{itemize}
			\item A patent is an exclusive right granted for an invention that is new, involves an inventive step, and is industrially applicable.
			\item The patent holder has the exclusive right to make, use, or sell the invention for 20 years.
			\item To obtain a patent, an application must be filed with the patent office, disclosing the invention.
		\end{itemize}
		\vspace{0.5cm}
		\textbf{Types of Patents:}
		\begin{itemize}
			\item Utility patents: Protect new inventions or functional improvements.
			\item Design patents: Protect the appearance of a product.
			\item Plant patents: Protect new varieties of plants.
		\end{itemize}
	\end{frame}
	
	% Slide 5: Trademarks
	\section{Trademarks}
	\begin{frame}{What is a Trademark?}
		\begin{itemize}
			\item A trademark is a recognizable sign, design, or expression that identifies products or services of a particular source.
			\item Trademark protection prevents others from using similar marks that may confuse consumers.
			\item Trademarks are registered through the national trademark office and can be renewed indefinitely every 10 years.
		\end{itemize}
		\vspace{0.5cm}
		\textbf{Examples:}
		\begin{itemize}
			\item Company logos, product names, slogans.
		\end{itemize}
	\end{frame}
	
	% Slide 6: Trade Secrets
	\section{Trade Secrets}
	\begin{frame}{What is a Trade Secret?}
		\begin{itemize}
			\item A trade secret is any confidential business information that provides a competitive edge.
			\item Trade secrets are not registered like patents or trademarks, but they are protected through non-disclosure agreements and other legal frameworks.
			\item To qualify as a trade secret, the information must be:
			\begin{itemize}
				\item Secret (not public knowledge),
				\item Valuable because it is secret,
				\item Subject to reasonable steps to keep it secret.
			\end{itemize}
		\end{itemize}
		\vspace{0.5cm}
		\textbf{Examples:}
		\begin{itemize}
			\item Coca-Cola formula, customer lists, marketing strategies.
		\end{itemize}
	\end{frame}
	
	% Slide 7: IPR Protection Mechanisms
	\section{IPR Protection Mechanisms}
	\begin{frame}{IPR Protection Mechanisms}
		\begin{itemize}
			\item \textbf{Copyrights:} Automatically protected upon creation, but can be registered for additional benefits.
			\item \textbf{Patents:} Requires filing an application with the national patent office.
			\item \textbf{Trademarks:} Requires filing an application with the trademark office and regular renewal.
			\item \textbf{Trade Secrets:} Protected by keeping the information confidential and using legal agreements like NDAs.
		\end{itemize}
	\end{frame}
	
	% Slide 8: International IPR Framework
	\section{International IPR Framework}
	\begin{frame}{International IPR Framework}
		\begin{itemize}
			\item The World Intellectual Property Organization (WIPO) oversees global IPR laws and treaties.
			\item Major treaties include:
			\begin{itemize}
				\item Berne Convention for the Protection of Literary and Artistic Works (copyrights),
				\item Paris Convention for the Protection of Industrial Property (patents and trademarks),
				\item TRIPS Agreement under the WTO.
			\end{itemize}
		\end{itemize}
	\end{frame}
	
	% Slide 9: Conclusion
	\section{Conclusion}
	\begin{frame}{Conclusion}
		\begin{itemize}
			\item Intellectual Property Rights are essential for encouraging innovation and protecting creators' rights.
			\item Understanding the different types of IPR and the mechanisms for protection can help individuals and businesses safeguard their creative works.
			\item Compliance with international frameworks ensures that creators are protected worldwide.
		\end{itemize}
	\end{frame}
	
	% Slide 10: Thank You
	\begin{frame}
		\begin{center}
			\Huge{Thank You!}
		\end{center}
	\end{frame}
	
\end{document}
