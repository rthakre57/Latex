%Here are 10 mathematical formulas written in LaTeX along with explanations. These formulas will be displayed using the standard LaTeX math mode and can be compiled into a PDF 
\documentclass{article}
\usepackage{amsmath}  % Package for advanced math

\title{10 Mathematical Formulas in LaTeX}
\author{Your Name}
\date{\today}

\begin{document}
	
	\maketitle
	
	Here are ten different mathematical formulas written in LaTeX:
	
	\section{Formulas}
	
	\subsection{1. Quadratic Formula}
	The solution to the quadratic equation \(ax^2 + bx + c = 0\) is given by:
	\[
	x = \frac{-b \pm \sqrt{b^2 - 4ac}}{2a}
	\]
	
	\subsection{2. Pythagorean Theorem}
	In a right triangle, the square of the hypotenuse is equal to the sum of the squares of the other two sides:
	\[
	a^2 + b^2 = c^2
	\]
	
	\subsection{3. Euler's Formula}
	Euler's famous equation, which relates the exponential function to trigonometric functions:
	\[
	e^{i\pi} + 1 = 0
	\]
	
	\subsection{4. Binomial Theorem}
	The binomial expansion for \((x + y)^n\) is:
	\[
	(x + y)^n = \sum_{k=0}^{n} \binom{n}{k} x^{n-k} y^k
	\]
	
	\subsection{5. Area of a Circle}
	The area \(A\) of a circle with radius \(r\) is:
	\[
	A = \pi r^2
	\]
	
	\subsection{6. Derivative of a Function}
	The derivative of a function \(f(x)\) is defined as:
	\[
	f'(x) = \lim_{\Delta x \to 0} \frac{f(x+\Delta x) - f(x)}{\Delta x}
	\]
	
	\subsection{7. Integral of a Function}
	The indefinite integral of a function \(f(x)\) is:
	\[
	\int f(x) \, dx = F(x) + C
	\]
	where \(F(x)\) is the antiderivative of \(f(x)\) and \(C\) is the constant of integration.
	
	\subsection{8. Taylor Series}
	The Taylor series expansion of a function \(f(x)\) around \(x = a\) is:
	\[
	f(x) = \sum_{n=0}^{\infty} \frac{f^{(n)}(a)}{n!} (x - a)^n
	\]
	
	\subsection{9. Probability of an Event}
	The probability \(P\) of an event \(A\) is given by:
	\[
	P(A) = \frac{\text{Number of favorable outcomes}}{\text{Total number of outcomes}}
	\]
	
	\subsection{10. Fundamental Theorem of Calculus}
	The fundamental theorem of calculus links differentiation and integration:
	\[
	\frac{d}{dx} \left( \int_{a}^{x} f(t) \, dt \right) = f(x)
	\]
	
\end{document}
